% Options for packages loaded elsewhere
\PassOptionsToPackage{unicode}{hyperref}
\PassOptionsToPackage{hyphens}{url}
\PassOptionsToPackage{dvipsnames,svgnames,x11names}{xcolor}
%
\documentclass[
  letterpaper,
  DIV=11,
  numbers=noendperiod]{scrartcl}

\usepackage{amsmath,amssymb}
\usepackage{lmodern}
\usepackage{iftex}
\ifPDFTeX
  \usepackage[T1]{fontenc}
  \usepackage[utf8]{inputenc}
  \usepackage{textcomp} % provide euro and other symbols
\else % if luatex or xetex
  \usepackage{unicode-math}
  \defaultfontfeatures{Scale=MatchLowercase}
  \defaultfontfeatures[\rmfamily]{Ligatures=TeX,Scale=1}
\fi
% Use upquote if available, for straight quotes in verbatim environments
\IfFileExists{upquote.sty}{\usepackage{upquote}}{}
\IfFileExists{microtype.sty}{% use microtype if available
  \usepackage[]{microtype}
  \UseMicrotypeSet[protrusion]{basicmath} % disable protrusion for tt fonts
}{}
\makeatletter
\@ifundefined{KOMAClassName}{% if non-KOMA class
  \IfFileExists{parskip.sty}{%
    \usepackage{parskip}
  }{% else
    \setlength{\parindent}{0pt}
    \setlength{\parskip}{6pt plus 2pt minus 1pt}}
}{% if KOMA class
  \KOMAoptions{parskip=half}}
\makeatother
\usepackage{xcolor}
\setlength{\emergencystretch}{3em} % prevent overfull lines
\setcounter{secnumdepth}{-\maxdimen} % remove section numbering
% Make \paragraph and \subparagraph free-standing
\ifx\paragraph\undefined\else
  \let\oldparagraph\paragraph
  \renewcommand{\paragraph}[1]{\oldparagraph{#1}\mbox{}}
\fi
\ifx\subparagraph\undefined\else
  \let\oldsubparagraph\subparagraph
  \renewcommand{\subparagraph}[1]{\oldsubparagraph{#1}\mbox{}}
\fi


\providecommand{\tightlist}{%
  \setlength{\itemsep}{0pt}\setlength{\parskip}{0pt}}\usepackage{longtable,booktabs,array}
\usepackage{calc} % for calculating minipage widths
% Correct order of tables after \paragraph or \subparagraph
\usepackage{etoolbox}
\makeatletter
\patchcmd\longtable{\par}{\if@noskipsec\mbox{}\fi\par}{}{}
\makeatother
% Allow footnotes in longtable head/foot
\IfFileExists{footnotehyper.sty}{\usepackage{footnotehyper}}{\usepackage{footnote}}
\makesavenoteenv{longtable}
\usepackage{graphicx}
\makeatletter
\def\maxwidth{\ifdim\Gin@nat@width>\linewidth\linewidth\else\Gin@nat@width\fi}
\def\maxheight{\ifdim\Gin@nat@height>\textheight\textheight\else\Gin@nat@height\fi}
\makeatother
% Scale images if necessary, so that they will not overflow the page
% margins by default, and it is still possible to overwrite the defaults
% using explicit options in \includegraphics[width, height, ...]{}
\setkeys{Gin}{width=\maxwidth,height=\maxheight,keepaspectratio}
% Set default figure placement to htbp
\makeatletter
\def\fps@figure{htbp}
\makeatother

\KOMAoption{captions}{tableheading}
\makeatletter
\makeatother
\makeatletter
\makeatother
\makeatletter
\@ifpackageloaded{caption}{}{\usepackage{caption}}
\AtBeginDocument{%
\ifdefined\contentsname
  \renewcommand*\contentsname{Table of contents}
\else
  \newcommand\contentsname{Table of contents}
\fi
\ifdefined\listfigurename
  \renewcommand*\listfigurename{List of Figures}
\else
  \newcommand\listfigurename{List of Figures}
\fi
\ifdefined\listtablename
  \renewcommand*\listtablename{List of Tables}
\else
  \newcommand\listtablename{List of Tables}
\fi
\ifdefined\figurename
  \renewcommand*\figurename{Figure}
\else
  \newcommand\figurename{Figure}
\fi
\ifdefined\tablename
  \renewcommand*\tablename{Table}
\else
  \newcommand\tablename{Table}
\fi
}
\@ifpackageloaded{float}{}{\usepackage{float}}
\floatstyle{ruled}
\@ifundefined{c@chapter}{\newfloat{codelisting}{h}{lop}}{\newfloat{codelisting}{h}{lop}[chapter]}
\floatname{codelisting}{Listing}
\newcommand*\listoflistings{\listof{codelisting}{List of Listings}}
\makeatother
\makeatletter
\@ifpackageloaded{caption}{}{\usepackage{caption}}
\@ifpackageloaded{subcaption}{}{\usepackage{subcaption}}
\makeatother
\makeatletter
\@ifpackageloaded{tcolorbox}{}{\usepackage[many]{tcolorbox}}
\makeatother
\makeatletter
\@ifundefined{shadecolor}{\definecolor{shadecolor}{rgb}{.97, .97, .97}}
\makeatother
\makeatletter
\makeatother
\ifLuaTeX
  \usepackage{selnolig}  % disable illegal ligatures
\fi
\IfFileExists{bookmark.sty}{\usepackage{bookmark}}{\usepackage{hyperref}}
\IfFileExists{xurl.sty}{\usepackage{xurl}}{} % add URL line breaks if available
\urlstyle{same} % disable monospaced font for URLs
\hypersetup{
  pdftitle={Final Grade Reflection},
  pdfauthor={Caleb Jensen},
  colorlinks=true,
  linkcolor={blue},
  filecolor={Maroon},
  citecolor={Blue},
  urlcolor={Blue},
  pdfcreator={LaTeX via pandoc}}

\title{Final Grade Reflection}
\author{Caleb Jensen}
\date{}

\begin{document}
\maketitle
\ifdefined\Shaded\renewenvironment{Shaded}{\begin{tcolorbox}[interior hidden, enhanced, frame hidden, boxrule=0pt, borderline west={3pt}{0pt}{shadecolor}, breakable, sharp corners]}{\end{tcolorbox}}\fi

\hypertarget{final-grade-reflection}{%
\subsection{Final Grade Reflection}\label{final-grade-reflection}}

After looking over my work and considering how the first half of this
class has gone I believe I have earned an A-. Starting by looking at
learning targets for the class I believe I have shown proficiency in
most if not all to this point. In the lab 4 challenge I showed that I
could import data from multiple file types, both .csv and .xls. In that
same challenge, looking at house prices and avocado sales, I also showed
I could select columns, filter rows or observations across numeric data
and in lab 3 I demonstrated the ability to filter across character data
types. Additionally in Lab 3 I was able to use mutate to modify
variables and create new ones, using if\_else to simplify a variable
from many groups to two. I was also able to use pivot\_longer() in Lab
4's challenge to create variables for city and move column names to be
observations. In that same challenge I used a mutating join
(left\_join()) to put a housing data set together with a data set that
looked at avocado prices and sales to explore the relationship between
the two. In that lab I also used a filtering join (semi\_join()) to
filter the data to include only the observations of cities that I was
interested in. In lab 4 I used a similar operation, an anti\_join() to
create data sets that held only certain cities or regions for further
exploration. I have been working on creating more professional appearing
labs and I believe my rendered documents have recently been appearing
much better than previously. Originally I had included output from
reading in data and calling packages. I have since learned to use
``\#\textbar{} output: FALSE'' properly to clean up the appearance of my
end product as can be seen in all my labs and challenges since lab 2,
and in the revised versions of the early ones as well. I attempt to
write well documented code and name many of my arguments so that each
line is easily understood and the goal of each input in a line is
understood. This can be seen in lab 4 challenge where I included things
like ``.cols'' in my separate function, ``by ='' in my filtering and
mutating join, or ``mapping ='' and ``x ='' in my ggplots to make them
easily read and understood. I also make my programs fairly dynamic by
using filtering joins such as in lab 4, or by using dynamic inputs that
can easily be changed without damage to the code itself. I have worked
on making visualizations for many data types. One example is lab ~where
I was able to recreate a bar chart, classified by region, a character
variable, using a proportion created from two numeric variables, and
colored by size, another character variable. In lab 2 I was able to use
character variables to create side by side box plots, overlaid with
jittered points. I was also able to color them by type and annotate the
graph itself making it easy to visually distinguish specific species.
This showed some creativity as well as demonstrating continued learning
by attempting a more difficult challenge successfully. I can also
summarize variables individually and across groups as seen in lab 3
where I was able to create summaries of both factor as well as numerical
variables about various demographic variables. Finally, I was able to
grow the efficiency of my code greatly by using the tidyverse and its
many related packages in every lab. I have become much more adept at
using pipelines, at pivoting data wide and long as seen in lab 4's
challenge, and at using modern filter and mutate techniques. My code
rarely repeats itself anymore. For example, I did a very poor job
originally in my lab 3 challenge, using about 8 statements where 1 was
needed, but I was able to fix the problem by including an across
statement and eliminating the excess statements. This shows the
increasing efficiency in my code, that I am attempting to maintain an
open mind by fixing and learning from my many errors, and that I can use
modern tools to iterate across columns and rows to reduce the repetition
and number of lines in my code. I have attempted to be an active and
helpful teammate throughout the quarter, asking for help when I need it
and offering it when others do and have really enjoyed working in my
group so far. I have also improved my peer review skills in helping
others make their code tidier, with proper line breaks and easy
readability as well as eliminating or reducing inefficient code and
replacing it with more modern techniques. I am trying to find the
balance between helpful feedback and building others up for their good
work. ~For instance, In a lab 4 peer review I recommended use of
filtering joins rather than the \%in\% operator, but mentioned that
using that was creative. Overall I would argue that I have put effort
into demonstrating proficiency and continued understanding in this
class. I have not always been perfect and there are many areas for
improvement, but on the whole I believe I have earned an A-.



\end{document}
